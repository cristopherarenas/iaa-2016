\section{Parámetros del algoritmo}

El algoritmo considera los siguientes parámetros:

\begin{itemize}
\item \generaciones: criterio de término para el algoritmo inmune. Es la cantidad de iteraciones que deben ocurrir para que finalice el algoritmo. El rango de valores permitidos para este parámetro va de 0 a infinito.
\item \alp{ } y \bet: representan variables de peso para ponderar las dos funciones objetivo y determinar una aptitud para una solución candidata. El rango de valores permitidos va de cero a uno. Además, la suma de ambas debe ser uno.
\item \popsize: representa el tamaño de la población $P$ en cada generación. Una generación está compuesta por un conjunto de soluciones candidatas factibles. El rango de valores permitidos va de 1 a infinito.
\item \clonsize: es el tamaño de la población de clones $C$ que se generarán en cada generación a partir de las mejores soluciones candidatas. El rango de valores permitidos va de 1 a infinito.
\item \pmejores: Porcentaje de mejores soluciones de la población que serán seleccionadas para mutarse en cada generación. El rango de valores permitidos va de 0 a 1.
\item \pclones: porcentaje de mejores clones mutados que serán almacenados en el conjunto de memoria $M$. El rango de valores permitidos va de 0 a 1.
\item \preemplazo: porcentaje de peores soluciones de la población que serán reemplazadas por soluciones aleatorias. El rango de valores permitidos va de 0 a 1.
\end{itemize}
