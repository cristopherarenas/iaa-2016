\section{Introducción}

Existen acercamientos que se encargan de resolver el Urban Transit Routing Problem (UTRP) \cite{metaheuristic2010}. Entre ellos se encuentran algoritmos evolutivos que emulan ciertos comportamientos de la naturaleza y los organismos de manera de encontrar soluciones a este tipo de problemas. \\

Dentro del grupo de los algorimos evolutivos, existen algunos que implementan ciertos comportamientos observados en los organismos de las personas. Los algoritmos inmunes artificiales representan una metáfora del sistema inmune donde se tiene un conjunto de anticuerpos que tienen la misión de encargarse de los virus y bacterias de enfermedades que atacan a un organismo. Los anticuerpos funcionan a través de procesos adaptativos, pues una vez que encuentran la mejor forma de curar una enfermedad se capacitan para futuras infecciones \cite{introduction} \cite{sia}. \\

En el presente documento se implementará un algoritmo evolutivo que se enfoca en simular el comportamiento del sistema inmune para resolver el UTRP. Posteriormete, se escogeran parámetros adecuados para el problema mediante un proceso de sintonización. Finalmente, se mostrarán algunos experimentos con instancias y las conclusiones del trabajo realizado.


