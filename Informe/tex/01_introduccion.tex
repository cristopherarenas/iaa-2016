\section{Introducción}

El diseño de un sistema de transporte que sea eficaz y eficiente es un problema importante en la actualidad y es de gran relevancia en las grandes urbes. porque implican translados de muchas personas en todo momento del día. Existen varios problemas asociados al diseño un buen sistema de transporte, los cuales se enfocan principalemente en establecer una buena planificación de horarios para conductores o buses, o definir enrutamientos que beneficien tanto a los usuarios como a las empresas operadoras de buses. El Urban Transit Routing Problem o UTRP \cite{metaheuristic2010} consiste en definir enrutamientos para una red de transporte público que sean capaces de reducir costos temporales tanto para pasajeros, como para operadores de buses. Es así como el UTRP se plantea como un problema multi-objetivo que requiere de técnicas que se encarguen de minimizar ambos costos.\\

En la literatura \cite{NewHaEOps,john2014improved} se han propuesto varios algoritmos, principalmente evolutivos, que tienen inspiración en fenómenos biológicos, para obtener soluciones en el UTRP. Los algoritmos propuestos establecen la evolución de soluciones iniciales definidas mediante estratégias o heurísticas, las cuales son modificadas mediante operadores de transformación, de manera de encontrar en pequeñas vecindades soluciones similares que puedan ser más beneficiosas para pasajeros u operadores de buses. Los Algoritmos Inmunes Artificiales o AIA \cite{introduction,sia} corresponden a una familia de algoritmos que se basan en el sistema inmune de un organismo, donde un conjunto de anticuerpos debe enfrentarse a un virus o infección. Los anticuerpos funcionan por medio de procesos adaptativos, pues una vez que encuentran la mejor forma de curar una enfermedad, se capacitan para tomar acción ante futuras infecciones.\\

En el presente trabajo considera una propuesta de AIA, con sus respectivos procesos de transformación y selección. Esto es, mediante la definición de operadores de transformación específicos para el UTRP y por medio de la presentación de parámetros relevantes para los procesos de selección. En el capítulo 2 se presentará el estado del arte del UTRP y se mostrarán enfoques y técnicas utilizadas para encontrar soluciones. El capítulo 3 describe en detalle el UTRP, mencionando las variables, parámetros y restricciones que considera el problema, asi como los objetivos que deben minimizarse. El capítulo 4 muestra la representación de elementos importantes para el problema. Los capítulos 5, 6, 7 y 8 presentan el AIA propuesto, detallando los pasos necesarios para generar soluciones factibles, parámetros asociados, operadores de tranformación y otras consideraciones importantes. El capítulo 9 detalla una serie de experimentos que se realizarán sobre instancias utilizadas en la literatura, cuyos resultados obtenidos se mostrarán en el capítulo 10.

