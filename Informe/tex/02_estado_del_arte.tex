\section{Estado del Arte}
El problema de enrutamiento de tr\'ansito urbano (UTRP por sus siglas en ingl\'es) involucra la ideaci\'on de rutas para el transporte p\'ublico.
Se trata de un problema NP-Duro altamente complejo, y resolverlo involucra invariablemente un ciclo de
generaci\'on y prueba de grupos de rutas candidatas. La mayor parte de literatura lo considera parte de un problema de escala mayor, el Urban Transit Network Design Problem (UTNDP por sus siglas),
el cual es dividido tanto en el UTRP como UTSP (Urban Transit Scheduling Problem), y tal como se puede diferir de sus nombres, el UTSP
tiene un enfoque de agendar los tiempos de llegada de los medios de transporte, mientras que el UTRP se enfoca en las rutas que estos
utilizan, ambos para mejorar el sistema de transporte urbano en las ciudades.\\
Los primeros acercamientos al UTRP lo tratan como un problema mono-objetivo. En~\cite{metaheuristic2010} se
comparan dos acercamientos de b\'usqueda local para resolver el problema. En este
caso se consideran dos objetivos: la distancia acumulada de todos los pasajeros
del bus y el n\'umero de trasbordos para la demanda completa. Se considera toda la demanda como satisfecha y el tiempo
promedio que cada usuario destina en viajar es
el menor posible. Adem\'as, se considera que cada ruta del conjunto est\'a libre de ciclos y retrocesos;
el conjunto de rutas est\'a conectado; hay exactamente $r$ rutas en el conjunto y el n\'umero de nodos en cada
ruta debe ser mayor a uno y no debe exceder el valor m\'aximo definido. \\
La funci\'on objetivo est\'a dada por la suma ponderada de ambos objetivos.
La inicializaci\'on se realiza de manera aleatoria respetando el largo establecido. Se
utiliza el movimiento \emph{Make-small-change}, que considera 3 posibilidades: Agregar un nodo en
la \'ultima posici\'on de la ruta, borrar el primer nodo de la ruta e invertir el orden de nodos en la ruta.
En~\cite{memetic2011} se propone un algoritmo mem\'etico cuyo objetivo es minimizar la suma del costo
para los usuarios y la demanda insatisfecha para la red de rutas. La inicializaci\'on se reliza
de manera aleatoria, se utiliza cruzamiento de rutas en un punto y mutaci\'on de una ruta por otra ruta factible. Para
la b\'usqueda local se seleccionan uno o dos cromosomas aletatoriamente, y se combinan de acuerdo a: (1) 
movimiento 2-opt, (2) intercambio de dos paraderos y (3) reubicaci\'on de una parada. Al final del proceso
se seleccionan los mejores $\mu$ cromosomas padre y $\lambda$ cromosomas hijo.
En~\cite{jiang2010improved} se utilizan colonias de hormigas para el UTRP. A diferencia de otros problemas resueltos con colonias de
hormigas, las hormigas no deben recorrer todos los nodos, sino que deben ir desde cierto nodo hasta otro.
Se toma como consideraci\'on que cada nodo podr\'a tener a lo m\'as 4 vecinos y las conexiones con estos ser\'an aquellas pertenceciente a
un conjunto de arcos permitidos. Si en alg\'un momento la ruta escogida por la hormiga no cumple
alguna de las restricciones (por ejemplo, ciclos en el recorrido), la hormiga se declara muerta y se castiga el
camino escogido.\\

Es posible encontrar acercamientos multi-objetivo para otros problemas similimares. Josefowiez, Semet y Talbi
en~\cite{vrprb} utilizan diversificaci\'on elitista y modelo de islas para el vehicle routing problem con
dos objetivos: minimizaci\'on del largo de las rutas y la diferencia entre el tama\~no de la ruta m\'as larga y la m\'as corta.
En este caso se utiliza ranking de dominancia para evaluar a los invididuos, donde los individuos no dominados de la poblaci\'on
forman el conjunto $E_1$ de ranking 1 y el resto de los individuos se agrupan en conjuntos $E_k$ donde cada elemento es dominado
por todos los elementos de los conjuntos anteriores.
En~\cite{ttvrp} se propone un MOEA h\'ibrido que incorpora b\'usqueda local y el concepto de optimalidad de Pareto para el
problema de encontrar una programaci\'on de rutas que cumpla con todas las entregas de una empresa de despacho, minimizando las distancias
recorridas y el n\'umero de camiones. Al igual que en el caso anterior se utiliza un ranking de fitness de Pareto para evaluar la calidad
de las soluciones. Zhang, Wang y Tang~\cite{events} investigan el problema de dise\~no de rutas para mega-eventos donde hay un gran
tr\'afico de transporte p\'ublico y se requieren rutas adicionales para el transporte de las personas que participan de un evento,
a su lugar de destino. En~\cite{events} se utiliza un algoritmo gen\'etico cuya funci\'on de evaluaci\'on es la distancia total del
conjunto de rutas. Adem\'as, se utilizan dos penalizaciones: la primera se aplica por cada nodo de destino que qued\'o fuera de ruta y
la segunda se aplica si la distancia total de una ruta supera la m\'axima distancia permitida.\\

Acercamientos multi-objetivos espec\'ificos para el UTRP se pueden encontrar en~\cite{zhang2010multi} donde se utiliza un algoritmo
evolutivo que usa el operador \emph{Make-small-change}. En este caso se utiliza el concepto de dominancia de Pareto para construir
el conjunto de soluciones. Este algoritmo entrega buen conjunto de rutas desde el punto de vista de los pasajeros y con mejores costos
para operadores en comparaci\'on a un algoritmo mono-objetivo similar con el que se compar\'o. Mumford en~\cite{NewHaEOps} utiliza exactamente
el mismo modelado, pero agrega un operador de cruzamiento al algoritmo evolutivo. El operador de cruzamiento selecciona intercaladamente
rutas de ambos padres de manera tal de tener una cantidad equitativa de rutas de cada padre en el hijo. En~\cite{GAUTRP}
se considera el mismo modelo y funci\'on objetivo que los anteriores, pero 
se realizan diferentes operaciones sobre un set de rutas para tratar de mejorar su calidad. Entre las modificaciones
se encuentran: una selecci\'on mediante ruleta de las rutas a modificar, una operaci\'on crossover entre dos rutas padres,
y la revisi\'on de factibilidad de los hijos, para finalmnente realizar una operaci\'on de mutaci\'on sobre dos rutas pertenecientes
al mismo set, intercambiando nodos aleatorios.\\

Para hacerse una idea de la complejidad real del problema es posible mencionar que el sistema de transportes de la ciudad de Santiago, Chile,
Transantiago cuenta con $372$ recorridos, y $11272$ paradas disponibles. Dichas paradas se distribuyen en siete unidades agrupadas por
zonas~\cite{datosGob}. Adem\'as, cuenta con 7 tipos de servicios entre los que se cuentan: servicios normales, cortos, expresos, variantes,
nocturnos, especiales e inyectados que operan en distintas condiciones horarias y de capacidad. Los estudios realizados en el caso
particular de Santiago coonsideran microsimulaciones de focos de congesti\'on considerando
diferentes escenarios de aumento de la capacidad vial, reversibilidad de las pistas y simulaci\'on de incidentes~\cite{alarcon2010}, 
mientras que en~\cite{cortes2009} se presenta un metodo de resoluci\'on para el problema de asignaci\'on de horarios, rutas y asignaci\'on
de choferes para una de las siete empresas del Transantiago (STP Santiago) que cuenta con aproximadamente 300 buses.
El autor aborda el problema mediante MIP (Programaci\'on Entera Mixta).
