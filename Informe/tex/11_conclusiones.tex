\section{Conclusiones}

El problema de enrutamiento de tránsito urbano, o UTRP en inglés, es un tipo de problema complejo y contingente en los tiempos actuales, ya que intenta satisfacer las necesidades de las personas y de las empresas que se encargan de administrar el transporte público. Este es un problema NP-duro que involucra diseñar un conjunto de rutas a partir de un sistema compuesto por paraderos y rutas que los conectan. Este problema ha sido tratado como un problema mono-objetivo y más recientemente como un problema multi-objetivo por medio de heurísticas y algoritmos evolutivos.

El Algoritmo Inmune Artificial (AIA) toma como base un sistema inmune donde un conjunto de anticuerpos debe enfrentarse a las infecciones que atacan a un organismo y además almacenan información para evitar nuevas infecciones futuras. El AIA propuesto en este trabajo plantea resolver el UTRP como un problema multi-objetivo, de manera de reducir tanto los costos de los pasajeros que utilizan el transporte público asi como los costos de los operadores que manejan el sistema de transporte y obteniendo por tanto un conjunto de soluciones. Los costos asociados que se plantean minimizar están cuantificados por tiempos, ya que la información que se tiene de este problema consiste principalmente en tiempos de desplazamiento entre paradas. Se plantea una inicialización de soluciones factibles, de manera de satisfacer con las restricciones impuestas en el problema. Luego por medio de tres operadores de mutación, se realizan cambios en solcuciones candidatas y se almacena una parte de las mejores soluciones en un conjunto de memoria. Parámetros asociados a los tamaños de poblacion y a selecciones de soluciones que pasarán a otra etapa son considerados en este algoritmo. Una propuesta interesante para este trabajo tiene relación con la aplicación de operadores de transformación de manera proporcional a la aptitud de una soliución candidata y la selección de mejores soluciones utilizando el ranking de dominancia.

Dos instancias conocidas en la literatura fueron utilizadas para encontrar los mejores parámetros en cada caso por medio de un proceso de sintonización. La primera de ellas, Mandl, tiende a descartar menos soluciones que la segunda, Mumford0, al obtener un mayor valor en el parámetro que selecciona soluciones para mutar. El conjunto de soluciones fue cuantificado mediante la métrica de hipervolumen y fue mostrado gráficamente en frentes de Pareto. Para ambas instncias, el conjunto de soluciones no parece verse afectado por la elección de números aleatorios y entregan valores de hipervolumen estables y en rangos definidos. Adicionalmente, la métrica de hipervolumen crece a medida que transcurren las generaciones. Sin embargo, los frentes de Pareto se comportan de distinta forma. Como Mumford0 es una instancia considerablemente más grande que Mandl, los tiempos de cómputo por semilla son 4 a 5 veces mayores. En la instancia Mandl, hubo una tendencia a generar muchas soluciones que benefician a operadores y pocas soluciones que benefician a pasajeros, mientras que en Mumford0 no se observó este fenómeno.

Al hacer un anásisis de las mejores soluciones para pasajeros, en ambas instancias se obtuvieron rutas extensas, que suponen una minimización para el tiempo de viaje promedio al evitar tener que hacer transbordos. En el caso de las mejores soluciones para operadores, las soluciones están conformadas mayormente por rutas cortas, lo que es consistente con la minimización de la función objetivo de operadores en alusión a minimizar el tiempo de viaje de todas las rutas. Al comparar las mejores soluciones con las cotas inferiores definidas para cada instancia, en Mandl se obtuvo una menor diferencia con estas cotas que en Mumford0. Un caso a notar es la mejor solución en Mandl para operadores, la cual no puede ser mejorada porque es la cota inferior de la instancia.

En la literatura se plantea un algoritmo evolutivo multi objetivo, MOEA, el cual es mejorado mediante una heurística que propone múltiples operadores de mutación. Si bien, el análisis no es exactamente comparable por las diferencias entre ambos acercamientos, el AIA obtuvo resultados cercanos a los planteados por esta técnica. Un aspecto a notar en la técnica MOEA de la literatura es el foco en mejorar las soluciones extremas del frente de Pareto y que el AIA propuesto no considera. Es por esto, que el AIA obtiene mejores valores de operadores en la mejor solución de pasajeros y viceversa.

Como trabajo a futuro, queda planteado un mejoramiento al AIA por medio de nuevos operadores de mutación, los que podrían ayudar a encontrar nuevos sectores inexplorados en el conjunto de soluciones. Por ejemplo, probar un operador que trabaje sobre dos rutas podría ser mejor que los tres operadores propuestos, aunque la factibilidad de que el operador sea exitoso y logre realizarse una tranformación podría ser menor.