\begin{abstract}
Los algoritmos inmunes artificiales (AIA) son estrategias de resolución de problemas evolutivos, lo que quiere decir que cambian con el tiempo y representan analogías de ciertos comportamientos biológicos. Estos algoritmos se puede utilizar en ciertos problemas, como por ejemplo el Urban Transit Routing Problem UTRP, en el cual se pretende encontrar rutas que satisfagan a operadores de los buses y a las personas que transportan. Por medio de un modelamiento, una representación del problema y la resolución por medio de un AIA se encontrará una solución que pueda satisfacer un conjunto de restricciones y trabajar con ciertos objetivos.
\end{abstract}
