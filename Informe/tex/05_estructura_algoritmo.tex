\section{Estructura del algoritmo}

El Algoritmo Inmune Artificial implementado está basado en un algoritmo de selección clonal genérico, propuesto por de Castro y Von Zuben \cite{read2012introduction}.

En el Algoritmo \ref{alg:aia} se muestran un pseudocódigo del algoritmo implementado. Un anticuerpo se entiende dentro del contexto de los algoritmos inmunes artificiales como una solución candidata. 

\begin{algorithm}
\caption{Algoritmo Inmune Artificial basado en Selección Clonal}\label{alg:aia}
\begin{algorithmic}[1]
\REQUIRE  Información del problema, información de instancia
\ENSURE Un conjunto de memoria $M$
\STATE Inicializar aleatoriamente población $P$ de tamaño \texttt{POP\_SIZE}
\STATE $g \leftarrow 1$
\WHILE{$g<=$\texttt{GENERACIONES}}
	\STATE Eliminar de la población $P$ anticuerpos dominados
	\FOR{\textbf{cada} $p$ perteneciente a la población $P$}
		\STATE Calcular aptitud de $p$
	\ENDFOR
	\STATE Seleccionar $mp$ de los mejores anticuerpos de $P$
	\STATE Generar un conjunto de clones $C$ de tamaño \texttt{CLON\_SIZE} con anticuerpos $mp$
	\FOR{\textbf{cada} clon $c$ perteneciente al conjunto $C$}
		\STATE $k \leftarrow$ numero de mutaciones a realizar de acuerdo a aptitud de $c$		
		\STATE Mutar $k$ veces el clon $c$
	\ENDFOR
	\STATE Copiar $mp$ clones a la población $P$
	\STATE Seleccionar $mm$ anticuerpos de $P$
	\STATE Copiar los anticuerpos $mm$ al conjunto de memoria $M$
	\STATE Reemplazar $pp$ anticuerpos con anticuerpos generados aleatoriamente
	\STATE $g \leftarrow g+1$
\ENDWHILE
\end{algorithmic}
\end{algorithm}
